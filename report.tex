\documentclass [11pt]{article}
\usepackage{amsfonts}
\usepackage{amsmath}
\usepackage{url}
\usepackage{amsthm}
\usepackage{times}
\usepackage{graphicx}
\usepackage{caption}
\usepackage{subcaption}
\usepackage{algpseudocode}
\usepackage{algorithm}
\usepackage{pdfpages}


\title{Stereo matching with the TRW-S Algorithm}
\author{Desmaison Alban \and Bunel Rudy}
\author{
  Desmaison, Alban\\
  \texttt{alban.desmaison@student.ecp.fr}
  \and
  Bunel Rudy\\
  \texttt{rbunel@ens-cachan.fr}
}

\begin{document}
\maketitle


\section{Principle}
The assignment consisted in solving a multi-label MRF problem. In order to solve it, we are going to use the TRW-S algorithm. Our problem is reducible to minimizing an energy on a graph.\\

Each node of this graph consists in a pixel. The unary potential that we have for each pixels correspond to the difference in intensity between the pixel in the left image and the pixel in the right image that our labels makes it correspond to. The weights in the edge enforce a smoothness in the label attribution: neighboring pixels should have similar labels, except if they belong ot different objects.


Energy minimization on a graph like this that contains loop is not efficiently doable. We are going to split the graph on which we are minimizing the energy in several trees in order to be able to efficiently minimize. This represents a dual of our problem, that we can solve approximately.

\section{Results}

\begin{figure}[htbp]
  \centering
  \includegraphics[width=0.6\textwidth]{labelsImage}
  \caption{Labels obtained in the end}
  \label{fig:labels}
\end{figure}

\begin{figure}[htbp]
  \centering
  \includegraphics[width=0.6\textwidth]{evolution}
  \caption{Evolution of the primal and dual}
  \label{fig:evolution}
\end{figure}

\end{document}